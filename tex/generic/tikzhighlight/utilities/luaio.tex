%-------------------------------------------------------------------------------
%
% luaio
%
% Provides a replacement for \newwrite that uses the Lua io module instead of
% the limited set of 16 TeX write registers. Requires LuaTeX.
%
% Copyright 2011 by Charlie Sharpsteen.
%
% This program is free software: you can redistribute it and/or modify
% it under the terms of the GNU General Public License as published by
% the Free Software Foundation, either version 3 of the License, or
% (at your option) any later version.
%
% This program is distributed in the hope that it will be useful,
% but WITHOUT ANY WARRANTY; without even the implied warranty of
% MERCHANTABILITY or FITNESS FOR A PARTICULAR PURPOSE.  See the
% GNU General Public License for more details.
%
% You should have received a copy of the GNU General Public License
% along with this program.  If not, see <http://www.gnu.org/licenses/>.
%
%-------------------------------------------------------------------------------

% TODO: Add proper @ name mangling to every thing in here. Currently everything
% is user-accessible to facilitate package testing. Move the \makeatletter
% operations into proper front end code.
\makeatletter

\newif\ifluaio@havelua

\ifx\directlua\undefined
  \luaio@haveluafalse
  \immediate\write16{No Lua}
\else
  \ifx\directlua\relax
    \luaio@haveluafalse
    \immediate\write16{No Lua}
  \else
    \luaio@haveluatrue
    \immediate\write16{Have Lua}
  \fi
\fi

\ifluaio@havelua
  % In most cases, LuaTeX only defines the \directlua primitive command. This
  % adds some additional TeX commands to our namespace by prefixing them with
  % luaio@.
  \directlua{ tex.enableprimitives('luaio@', {'luaescapestring', 'latelua'}) }

  % TODO: Properly namespace the Luamodulet defined below.
  \directlua{

    ltio = {}
    ltio.handles = {}

    function ltio.new_handle()
      table.insert(ltio.handles, 0)
      tex.print(table.getn(ltio.handles))
    end

    function ltio.open_handle(n, path)
      if not ltio.handles[n] == 0 then
        ltio.handles[n]:close()
      end

      ltio.handles[n] = io.open(path, "w")
    end

    function ltio.close_handle(n)
      if not ltio.handles[n] == 0 then
        ltio.handles[n]:close()
        ltio.handles[n] = 0
      end
    end

    function ltio.write_handle(n, string)
      if ltio.handles[n] == 0 then
        texio.write("\string\n" .. string .. "\string\n")
      else
        ltio.handles[n]:write(string .. "\string\n")
      end
    end

  }


  % Defined as a "outer mode only" macro so it can't be nested inside other macro
  % definitions. This matches the way the plain TeX macro \newwrite is defined.
  \outer\def\newlwrite#1{\edef#1{\directlua{ltio.new_handle()}}}

  \def\lopenout#1#2{\luaio@latelua{ltio.open_handle(#1, "\luaio@luaescapestring{#2}")}}
  \def\lcloseout#1{\luaio@latelua{ltio.close_handle(#1)}}
  \def\lwrite#1#2{\luaio@latelua{ltio.write_handle(#1, "\luaio@luaescapestring{#2}")}}
\fi

\makeatother
\endinput

